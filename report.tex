\documentclass[a4paper,11pt,twoside]{article}
%\documentclass[a4paper,11pt,twoside,se]{article}

\usepackage{UmUStudentReport}
\usepackage{verbatim}   % Multi-line comments using \begin{comment}
\usepackage{courier}    % Nicer fonts are used. (not necessary)
\usepackage{pslatex}    % Also nicer fonts. (not necessary)
\usepackage[pdftex]{graphicx}   % allows including pdf figures
\usepackage{listings}
\usepackage{pgf-umlcd}
\usepackage{blindtext}
\usepackage{rotating}
\usepackage{enumitem}
%\usepackage{lmodern}   % Optional fonts. (not necessary)
%\usepackage{tabularx}
%\usepackage{microtype} % Provides some typographic improvements over default settings
%\usepackage{placeins}  % For aligning images with \FloatBarrier
%\usepackage{booktabs}  % For nice-looking tables
%\usepackage{titlesec}  % More granular control of sections.

% DOCUMENT INFO
% =============
\department{Department of Computing Science}
\coursename{Operating Systems 7.5 p}
\coursecode{5DV171 HT17}
\title{Assignment 2, Linux Scheduler Benchmarking}
\author{Lorenz Gerber ({\tt{dv15lgr@cs.umu.se}})}
\date{2017-10-10}
%\revisiondate{2016-01-18}
\instructor{Jan Erik Moström / Adam Dahlgren Lindström}


% DOCUMENT SETTINGS
% =================
\bibliographystyle{plain}
%\bibliographystyle{ieee}
\pagestyle{fancy}
\raggedbottom
\setcounter{secnumdepth}{2}
\setcounter{tocdepth}{2}
%\graphicspath{{images/}}   %Path for images

\usepackage{float}
\floatstyle{ruled}
\newfloat{listing}{thp}{lop}
\floatname{listing}{Listing}



% DEFINES
% =======
%\newcommand{\mycommand}{<latex code>}

% DOCUMENT
% ========
\begin{document}
\lstset{language=C}
\maketitle
\thispagestyle{empty}
\newpage
\tableofcontents
\thispagestyle{empty}
\newpage

\clearpage
\pagenumbering{arabic}

\section{Introduction}
The aim was to benchmark two or more schedulers for the linux kernel.
For this purpose, a user-space program had to be written and obtained
data was to be evaluated using an analysis framework of choice.
Here,

\section{Material and Method}
\subsection{The benchmark program}
Sorting arrays of various length. Each array sorting one thread. Each thread
reports start time, end time and runtime. Tasks should take reasonably long.
Use a fixed number of threads. For example 50. Data creation ahead of
processing. Create one long array and copy to new arrays.

length of arrays in assignment order to threads:
increasing
decreasing
steady
random



\subsection{Data Evaluation}
Data evaluation by R. Same experiment run 10-20 times.
Barchart for 10, 10, 1, - array length, showing three bars for each array length (fifo, RR, CFS).
barchar with total run time for fifo, RR, CFS



\section{Results}


\section{Discussion}


\addcontentsline{toc}{section}{\refname}
%\bibliography{references}

\end{document}
